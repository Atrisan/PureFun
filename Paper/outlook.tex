\section{Conclusion/Future Work}
In this paper we introduced the imperative pure programming language PureFun we developed using MontiCore 5. We started by introducing our design choices in section 2, where we elaborated on the imperative syntax and why we decided to adapt the style of both Python and C++. In the second part of section 2 we discussed our choice for the generated C++ code. The section Pureness introduce 3 strategies to avoid side effects in PureFun for serial execution. Namely: immutable global variables, immutable data structures and call-by-value functions. Additionally, the third section also discussed a method to avoid data races by utilizing futures and promises in the generated C++ code. Section 4, covered the core features of PureFun. During the first subsection the container types: list, tuple and map got introduced. Afterwards, the type checking algorithm for PureFun was explained. In the end, the async feature, which can both be used to call functions as well as execute a block asynchronously. The section was concluded by a number of best practices for the async feature, to avoid unnecessary overhead.\\
In the future, we plan to add a number of new features as well as extend on already existing ones. This includes an improve type checking algorithm, that covers data structures, as well as optionals for data structures. Furthermore, we plan to introduce short hand notations for data structures. Most notably, extensions for the async feature are planned to extend upon parallel patterns and introduce GPU support.
A natural extension would be to generate efficient code in order to also leverage distributed systems. Similarly, we plan to employ a number of optimizations, such as smart thread creation, operation fusing etc. to make efficient use of computational resources.