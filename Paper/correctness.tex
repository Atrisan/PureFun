\section{Pureness}
In this section we discuss the foundation of PureFun, the pure functions. Pure functions do not induce any form of side effects during execution. To further elaborate on the term side effect we introduce the term idempotence.\\
A function is idempotent, if it can be called multiple times with the same arguments and always returns the same result. We define functions fulfilling  the criteria of idempotence as pure.\\
As an example for a non-pure function we take a look at a naive implementation of the length function for lists in Java. 
\begin{figure}[h]
\begin{lstlisting}[caption={Java code that deletes its input list as a side effect.},label={JavaEx}]
class List {
  Data value;
  List next;
  static int len (List l) {
    int n = 1;
    while (l.next != null) {
      l = l.next;
      n = n + 1;
    }
    return n;
  }
}
\end{lstlisting}
\end{figure}
In Listing \ref{JavaEx} we defined a new class List and the member function len, which returns the length of the list. Java utilizes references when using objects as arguments. Thus, the function len operates on the original and deletes all elements other than the last element of the list. The overwriting happens in line 8, where the list l becomes l.next. This results in a successive deletion of the first element until only one element remains. Since this function changes its environment a second call would result in a length of 1. Therefore, this function is not pure. The environment of a function is the assignment of all free variables in the function body. A function that changes these free variables may not be pure.\\
To prevent functions from changing their environment we embedded 3 strategies into PureFun:
\begin{enumerate}
\item Immutable global variables
\item Immutable data structures
\item Call-by-value functions
\end{enumerate}
\subsection{Immutable global variables}
\begin{figure}[h]
\begin{lstlisting}[caption={C++ code that uses a global variable.},label={C++Ex}]
int n = 0;
bool f() {
  if (n < 1) {
    n++;
    return true;
  }
  return false;
}
\end{lstlisting}
\end{figure}
Global variables in PureFun allow the declaration of constants. In other programming languages global variables can be access by any function. Since especially write accesses are problematic for pure functions. In listing \ref{C++Ex} the function \texttt{f} tests whether the global variable \texttt{n} is smaller than zero, if that is the case n is incremented and \texttt{f} returns true. The second execution of \texttt{f} will return false, since \texttt{n=1} now holds. Thus, f is not pure. By restricting the accesses on global variables to read-only impure functions similar to the code in listing \ref{C++Ex} become impossible.
\subsection{Immutable data structures}
Data structures can become the reason for side effects easily, as seen in listing \ref{JavaEx}. Even if the argument \texttt{l} was a local variable, the process in lines 6-9 would still be undesired behaviour. To prevent such an undesired behaviour data structures are also restricted to read-only access. Thus, recursive data structures, are enforced to be evaluated recursively and as such being covered by the next strategy.
\subsection{Call-by-value functions}
The last strategy utilizes a feature by C++ \cite{C++STD}, which uses copies of the argument and not a reference to the argument. The call-by-value feature in the generated C++ code ensures, that the argument both allows read and write access.\\
Using the call-by-value functions in combination with both the immutable global variables and data structures, it can be ensured, that no function is able to change their environment. Thus, leading to pure functions during serial execution.
\subsection{Async functions/blocks}
The async feature allows the asynchronous execution of a single function as well as a structured block. Since data races are a common error when regarding code execution with more than one execution unit, we need to introduce an additional strategy to prevent those. \\
A data race is an error in which two or more threads access the same memory with at least one write access and without proper synchronization. Thus, leading to potentially non-deterministic results. To prevent said data races we made use of C++ futures and promises, that are part of the standard library since C++17. By generating asynchronous code into said futures, the generated C++ code starts a new thread to execute the stated function/block. The promises on the other hand handle the synchronization. To properly leverage the synchronization it is tested before the code generation if any variable is write-accessed in parallel. During the next access of the variable the promise is generated to ensure, that the parallel execution has finished before continuing the execution.\\
With the generated futures and promises we prevent any data races and maintain the purity of the serial execution of the same code